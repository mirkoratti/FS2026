\section{Kinetische Gas-Theorie}

\subsection{Aequipartitionsgesetz}

\textbf{Mittlere kinetische Energie} 

\smallskip

Idealisierte Annahmen: 

\begin{itemize}
	\item Moleküle = Massenpunkte
	\item Keine (bzw.) elastische Zusammenstösse
	\item Keine Kräfte zwischen den Molekülen
	\item Elastischer Stoss gegen Wand 
	\item Alle Moleküle haben gleiche Geschwindigkeit
	\item $\frac{1}{6}$ aller Moleküle fliegen gegen eine einzelne Wand
\end{itemize}

\medskip


\begin{minipage}[c]{0.48\columnwidth}
	$$ \boxed{ \overline{E} = f \cdot \frac{k \cdot T}{2} } $$
\end{minipage}
\hfill
\begin{minipage}[c]{0.48\columnwidth}
	\begin{tabular}{ll}
		$f = 3$	& 1-atomiges Gas \\
		$f = 5$	& 2-atomiges Gas \\
		$f = 6$	& 3-atomiges Gas \\
	\end{tabular}
\end{minipage}

\medskip

\begin{tabular}{c l c}
	$\overline{E}$	& Mittlere kinetische Energie												& $[\overline{E}] = \joule$ 		\\
	$f$ 			& Freiheitsgrade 															& $[f] = 1$ 						\\
	$k$ 			& Boltzmann-Konstante $k = 1.381 \cdot 10^{-23} \, \frac{\joule}{\kelvin}$	& $[k] = \frac{\joule}{\kelvin}$	\\
	$T$ 			& \textbf{Absolute} Temperatur 												& $[T] = \kelvin$
\end{tabular}


\subsection{Geschwindigkeiten}

\subsubsection{Mittlere quadratische Geschwindigkeit $\bm{u}$}

\vspace{-0.2cm}

$$ \boxed{ u = \sqrt{\frac{3 \cdot k \cdot T}{m}} = \sqrt{ \frac{3 \cdot R \cdot T}{M}} } $$


\subsubsection{Mittlere Geschwindigkeit $\bm{\overline{v}}$}

\vspace{-0.2cm}

$$ \boxed{ \overline{v} = \sqrt{\frac{8 \cdot k \cdot T}{\pi \cdot m}} = \sqrt{ \frac{8 \cdot R \cdot T}{\pi \cdot M}} } $$


\subsubsection{Wahrscheinlichste Geschwindigkeit $\boldsymbol{v_0}$}

\vspace{-0.2cm}

$$ \boxed{ v_0 = \sqrt{\frac{2 \cdot k \cdot T}{m}} = \sqrt{ \frac{2 \cdot R \cdot T}{M}} } $$

\renewcommand{\arraystretch}{1.3}
\begin{tabular}{c l c}
	$k$ & Boltzmann-Konstante $k = 1.381 \cdot 10^{-23} \, \frac{\joule}{\kelvin}$	& $[k] = \frac{\joule}{\kelvin}$			\\
	$T$ & \textbf{Absolute} Temperatur 												& $[T] = \kelvin$ 							\\
	$m$ & Masse des Teilchens 														& $[m] = \kilogram$							\\
	$M$ & Mol-Masse 																& $[M] = \frac{\kilogram}{\mole}$			\\
	$R$	& Universelle Gaskonstante $R = 8.314 \, \frac{\joule}{\mole \, \kelvin}$ 	& $[R] = \frac{\joule}{\mole \, \kelvin}$ 	\\
\end{tabular}
\renewcommand{\arraystretch}{1}


\subsection{Maxwell-Boltzmann-Verteilung}

\vspace{-0.2cm}

$$ \boxed{ f(m, \, T, \, v) = \sqrt{\frac{2 \cdot m^3}{\pi \cdot k^3 \cdot T^3 }}  \cdot v^2 \cdot \e ^{- \frac{m \cdot v^2}{2 \cdot k \cdot T}} } $$


\begin{tabular}{c l c}
	$m$ & Masse des Teilchens 														& $[m] = \kilogram$ 				\\
	$k$ & Boltzmann-Konstante $k = 1.381 \cdot 10^{-23} \, \frac{\joule}{\kelvin}$	& $[k] = \frac{\joule}{\kelvin}$	\\
	$T$ & \textbf{Absolute} Temperatur 												& $[T] = \kelvin$ 					\\
	$v$ & Geschwindigkeit 															& $[v] = \frac{\meter}{\second}$
\end{tabular}


\subsection[Mittlere freie Weglänge]{Mittlere freie Weglänge $\boldsymbol{\overline{\lambda}}$}

Gibt an, um welche Strecke sich ein Molekül im Mittel bis zum nächsten Zusammenstoss fortbewegen kann.

$$ \boxed{ \overline{\lambda} = \frac{1}{\sqrt{2}} \cdot \frac{1}{n \cdot (\pi \cdot d^2 )} }  \qquad \text{mit Wirkungsquerschnitt } \sigma = \pi \cdot d^2$$

\begin{tabular}{c l c}
	$n$ & \crd{Molekül-Dichte} 	& $[n] = \frac{1}{\meter^3}$	\\	
	$d$ & Molekül-Durchmesser 	& $[d] = \meter^2$
\end{tabular}


\subsection{Dichtefunktion}

\begin{minipage}[c]{0.58\columnwidth}
	Verteilungsfunktion der mittleren, freien Weglänge 
\end{minipage}
\hfill
\begin{minipage}[c]{0.38\columnwidth}
	$$ \boxed{ f(x) = \frac{1}{\overline{\lambda}} \cdot \e^{- \frac{x}{\overline{\lambda}}}  } $$
\end{minipage}


\subsection{Transportvorgänge}

\subsubsection{Wärmeleitung}

Transport von \textbf{kinetischer Energie} (als Wärme wahrgenommen)

$$ \boxed{ j_Q = - \lambda_Q \cdot \frac{\mathrm{dT}}{\mathrm{dx}} \qquad \quad \lambda_Q = \frac{1}{6} \cdot n \cdot \overline{v} \cdot \overline{\lambda} \cdot f \cdot k }  $$

\medskip

\begin{minipage}[c]{0.48\columnwidth}
	\subsubsection{Diffusion}

	Transport von \textbf{Masse} 
	$$ \boxed{ j_D = -D \cdot \frac{\mathrm{dn}}{\mathrm{dx}} \qquad \quad  D = \frac{1}{3} \cdot \overline{v} \cdot \overline{\lambda} }  $$
\end{minipage}
\hfill
\begin{minipage}[c]{0.48\columnwidth}
	\subsubsection{Viskosität}

	Transport von \textbf{Impuls} 
	$$ \boxed{ \tau = - \eta \cdot \frac{\mathrm{dv}}{\mathrm{dx}} \qquad \quad  \eta = \frac{1}{3} \cdot n \cdot \overline{v} \cdot \overline{\lambda} \cdot \rho }  $$
\end{minipage}

\medskip

\renewcommand{\arraystretch}{1.3}
\begin{tabular}{c l c}
	$j_Q$ 					& Wärmestrom 																& $[j_Q] = \frac{\watt}{\meter^2}$ 			\\
	$\lambda_Q$ 			& Wärmeleitfähigkeit 														& $[\lambda_Q] = \frac{\watt}{\kelvin}$ 	\\
	$j_D$ 					& Diffusionsstrom 															& $[j_D] = ?$ 								\\
	$D$ 					& Diffusionskonstante 														& $[D] = \frac{\meter^2}{\second}$ 			\\
	$\tau$ 					& Schubspannung 															& $[\tau] = \newton$ 						\\
	$\eta$ 					& Viskosität	 															& $[\eta] = \pascal \, \second$ 			\\
	$n$ 					& \crd{Molekül-Dichte} 														& $[n] = \frac{1}{\meter^3}$				\\	
	$\overline{v}$ 			& Mittlere Geschwindigkeit 													& $[\overline{v}] = \frac{\meter}{\second}$	\\
	$\overline{\lambda}$ 	& Mittlere freie Weglänge 													& $[\overline{\lambda}] = \meter$ 			\\
	$f$ 					& Anzahl Freiheitsgrade 													& $[f] = 1$ 								\\
	$k$ 					& Boltzmann-Konstante $k = 1.381 \cdot 10^{-23} \, \frac{\joule}{\kelvin}$	& $[k] = \frac{\joule}{\kelvin}$			\\
	$T$ 					& \textbf{Absolute} Temperatur 												& $[T] = \kelvin$ 							\\
	$\rho$ 					& Dichte 																	& $[\rho] = \frac{\kilogram}{\meter^3}$
\end{tabular}
\renewcommand{\arraystretch}{1}

